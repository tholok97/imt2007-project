% TODO: 
% * add prefix to \refs. "section \ref{..}" for example
% * branch site / branch location <- make consistent
% * make sure sections about HQ and branch appear in a consistent order :D *muh consistency*
% * Fix HQ vs Head Quarters
% * fix up ~ before \ref and \cite
% * fix organisation/organization
% * give lab access switches special treatment?
% * do we want to clearpage before \section's?
% * Phyical security!
% * centre vs center
% * clean up \\\\'s to look good and uniform
% * call our PT implementation a demo?
% * Run spellchecker
% * formatting for IP addresses? :/
% * replace "'S" with "'s"
% * branch -> learning centre
% * Gjøre guests større?
% * VLANs to VLAN's and so on
% * Make it clear that we are focusing on saving costs. Redundancy at the HQ for example is a luxury
% * Setup a pingable host in the services segment to use for demonstration
% * Write about firewalls, defence-in-depth, etc. in ICT discussion

% command to make notes to us in the document
\newcommand{\notetous}[1] {
  \textbf{\textit{\textcolor{red}{#1}}}
}

\input{preamble.tex}

\title{Charity Organization - "SikreNorge"}

% TODO: replace 923 with actual student number
\author{
  Job Nestor Bahner, 494300\\
  Johannes Borgen, 923\\
  Abdisalan Hussein, 923\\
  Sam Roen, 923\\
  Thomas Løkkeborg, 923
}

\begin{document}

\maketitle

\begin{abstract}
SikreNorge is a non-profit charity organization with locations all over Norway. In this report we will discuss how we would implement a network infrastructure for the organization.
\end{abstract}

\thispagestyle{empty}

\clearpage
\pagenumbering{roman}
\setcounter{page}{1}
\tableofcontents

\clearpage
\pagenumbering{arabic}

% BEGINNING OF CONTENT

\section{Business case}

SikreNorge is a non-profit charity organization that provides information security services to businesses and institutions in Norway. The organization seeks to secure Norway by helping those who would not otherwise have the means to prioritize security.
\\
\\
The organization is reliant on local expertise in each county in the form of volunteers and hired professionals. Volunteers are expected and encouraged to gain security competence through the experts. Good candidates for volunteers are students doing technical studies. Where possible the organization will make agreements with local universities teaching information technology courses to encourage students to volunteer. The organization is also open to volunteer work as part of theses and course projects. Volunteers of the organization do work in exchange for practical information security experience. Both volunteers and hired professionals are expected to sign strict non-disclosure agreements regarding their work with businesses and institutions.
\\
\\
Financially SikreNorge relies on donations, government subsidies, and income from their web-shop. Businesses who donate are encouraged to publicise their contribution to show that they support the organization's mission. This way both the contributing businesses and SikreNorge get good PR. Government subsidies are given with the intention and expectation that the organization will use these means to support local public and private institutions with information security tasks.

\subsection{Locations}

SikreNorge will establish learning centers in each of Norway's 19 counties. The headquarter lies in Oslo.

% TODO note that the number of counties is changing
Learning centers are meant as a hub for all the organizations activities in the county. The centers have offices for local management and hired professionals. They also contain network sandbox areas for practicing network security, classrooms for security courses, lecture halls for seminars and general storage space. In addition to activities held at the centers, the organization can send consultants into the field to do evaluation or configuration on-site. Consultants are a mix of local hired professionals, volunteers and staff.
\\
\\
The organization's headquarter provides the same services as learning centers, but also does the heavy lifting in terms of information technology services and infrastructure. The website, E-mail, management applications, and all the large equipment to which one can connect from another center for data resources are examples of services that are hosted here.

\subsection{Services}

The organization includes a wide range of services, and the network must be set up to support these. Following is a brief discussion of key services, along with some infrastructure concerns.

% TODO more services?

\subsubsection{On-site company consultation}

Volunteers, hired professionals and staff  can be sent out to do consulting on-site at businesses or institutions. Access to the organizations internal services might be required during consultation, so the consultants should be able to reach them from on-site. Confidential information might be sent and received during this communication, so the technology chosen for this should focus on security.

\subsubsection{Teaching at learning centre}

Businesses and institutions may visit a learning centre to take courses or to receive guidance. During this stay they will naturally want internet connectivity. In addition, access to some internal services might be needed for training purposes. A visitors access rights should be restricted as much as possible.

\subsubsection{Network security laboratory}

Each learning centre has a laboratory where visitors can explore hardware and software to learn about vulnerabilities and best practises. Labs are meant to be a sandbox area for learning about information security, so they should be separated from the rest of the distribution system in such a way that experiments from the lab cannot interfere with the rest of the learning centers operations.

% TODO "separated from the rest of the distribution system" - currently this is not done

\subsubsection{Public website}

SikreNorge's website includes information about the organizations offerings, online courses and a web-shop. Some of the website's information and courses might build upon internal services, so the infrastructure should support communication to allow this. The web-shop will sell T-shirts, coffee mugs, posters and more to further information security awareness and good practise. The shop allows buyers to register accounts for easy shopping.

\subsubsection{Management website}

The organization uses a website separate to the public website to manage internal logistics. Registered staff, businesses, institutions and overview of future work are examples of information that can be found here.

% TODO: refer do the bave and below sections in the ict implementation discussion

\section{Security Policy}

% NOTE BY THOMAS: This list was slightly changed from the one in docs, as LaTeX struggles with lists deeper than 4 levels :p

\subsection{Statement of policy}

\begin{enumerate}
  \item Scope and applicability
  \begin{enumerate}
    \item Our organization
      \begin{enumerate}
        \item Every member is expected to know the content of our policy and comply. Ignorance is equal to non compliance.
      \end{enumerate}
        \item Partners
      \begin{enumerate}
        \item They are expected to learn and know the content, whereas an equal understanding of security is key.
      \end{enumerate}
        \item Guests/Trainees
      \begin{enumerate}
        \item Must know key elements to ensure safe learning and usage when using our equipment. Coordinators must brief them before actual usage.
      \end{enumerate}
  \end{enumerate}
\end{enumerate}

\subsection{Responsibilities}

\begin{enumerate}
  \item CISO 
  \begin{enumerate}
    \item Controls the overall structure of security and security management.
    \item Must adhere to security guidelines and practice consistently, as they are a possible single-point-of-failure in the security scheme.
    \item Must have IT-Security background, in conjunction with management background.
  \end{enumerate}
  \item Branch office security managers
  \begin{enumerate}
     \item Manage security for their branch office.
     \item Report to CISO.
  \end{enumerate}
  \item Employees and volunteers
  \begin{enumerate}
    \item Be loyal to the organization's security guidelines.
    \item Understand Policy.
    \item Validate understanding of policy.
    \item Understand repercussions of policy insubordination.
  \end{enumerate}
\end{enumerate}

\subsection{Authorized access}

\begin{enumerate}
  \item User access
  \begin{enumerate}
    \item Administration, management and other vital parts of the organizational systems must be separated from normal users.
    \item Employees have a level of clearance, and access thereafter. 
    \item CISO have full access to systems, except those classified as secret.
    \item Employees with access to confidential information or higher must sign a non-disclosure agreement.
    \item Volunteers may have access to certain confidential information to perform their work, but their access is defined along the lines of a need to know basis.
    \item Levels of classification, where higher levels have access of the lower:
    \begin{enumerate}
      \item Official - All can access.
      \item Confidential - Employees, volunteers
      \item Restricted - Managers, CISO
      \item Secret - Executives
    \end{enumerate}
  \end{enumerate}
  \item Fair and responsible use
  \begin{enumerate}
    \item As a precaution, emails are to be treated critically.
    \begin{enumerate}
      \item Hyperlinks must not be clicked, instead copied and pasted into the search bar.
      \item Images not to be loaded before sender is verified.
      \item Autoload should remain off by default.
      \item Attachments not to be downloaded or viewed unless verified sender.
      \item The email server must have DMARC, SPF and DKIM.
    \end{enumerate}
    \item Passwords should have a certain standard.
    \begin{enumerate}
      \item Least 10 characters.
      \item Non dictionary words or other easily identifiable words.
      \item Mix of characters (lower-, and uppercase), numbers and symbols.
      \item Separate from personal passwords.
      \item Password manager on organization equipment encouraged. Required for individuals with high clearance.
    \end{enumerate}
    \item VPN required off-site.
  \end{enumerate}
  \item Protection of privacy
  \begin{enumerate}
    \item Must adhere to GDPR regulations.
    \item Must also comply with regulations from Datatilsynet.
    \item Must comply with Norwegian and international laws
  \end{enumerate}
\end{enumerate}

\subsection{Prohibited usage of equipment}

\begin{enumerate}
  \item Disruptive use and misuse
  \begin{enumerate}
    \item No personal equipment in restricted areas.
    \item USB devices must only be issued by the organization and only these can be used on organization equipment.
    \item Vital organizational equipment should not be brought off-site.
    \item Restricted equipment must not be used for not-intended purposes.
    \begin{enumerate}
      \item Web surfing.
      \item email.
      \item Farming(Unless intended).
      \item Hosting(Unless intended).
      \item Packet sniffing(Wireshark, unless intended).
    \end{enumerate}
    \item Users with high level of access must not use their accounts to perform non-vital and possibly compromising tasks.
    \begin{enumerate}
        \item Web surfing.
        \item Use email.
    \end{enumerate}
  \end{enumerate}
  \item Criminal use
  \item Offensive or harassing materials
  \item Copyrighted, licensed, or other intellectual property
\end{enumerate}

\subsection{Systems management}

\begin{enumerate}
  \item Management of stored materials
  \begin{enumerate}
    \item Authentication and authorization should be managed by an AAA server.
    \item All access to restricted equipment and actions must be logged on a server.
    \item Classified data need to be encrypted on the hard drive and in transfer - Partner and organizational data protected by non-disclosure contracts.
    \item All organization equipment that have been used to store organization or partner information must be disposed of in a safe manner.
    \item Hard drives must be destroyed by the organization itself.
    \item Printers must be in an enclosed network, and classified documents should only be printed on secure organization printers.
  \end{enumerate}
  \item Documentation
  \begin{enumerate}
      \item Important assets and equipment must be properly documented for the purpose of maintenance and operations.
      \item All configurations must be stored on a safe server for easy system recovery.
  \end{enumerate}
  \item Employer monitoring
  \begin{enumerate}
    \item Use of internet logged.
    \item Organizational equipment inventory checked regularly.
    \item Restricted area entry logged.
    \begin{enumerate}
      \item Physical - ID card entry log.
      \item Digital - Credentials log.
    \end{enumerate}
  \end{enumerate}
  \item Virus protection
  \begin{enumerate}
    \item Each branch must have its own firewall mechanism to filter unwanted traffic.
    \item Organization computers and smart-phones with internet connection must have anti-virus.
    \item Email etiquette, see "Fair and responsible use (a)".
  \end{enumerate}
  \item Physical security
  \begin{enumerate}
    \item Limited access to restricted areas.
    \begin{enumerate}
      \item Must use ID-card to enter: Server rooms and Offices for high clearance individuals.
    \end{enumerate}
    \item Guest premises must be separated from working grounds and other vital instances.
  \end{enumerate}
  \item Encryption
  \begin{enumerate}
    \item Within the organization the use of asymmetric encryption with security level equal to SHA 3 or higher must be enforced.
  \end{enumerate}
\end{enumerate}

\subsection{Violation of policy}

\begin{enumerate}
  \item Procedures for reporting violations
  \begin{enumerate}
    \item As a security pushing organization, reporting violations of our policy is encouraged.
    \item To ensure this practice the report can be submitted anonymously.
  \end{enumerate}
    \item Penalties for violations
  \begin{enumerate}
    \item Minor violations can result in revoked clearance or access.
    \item Serious violations:
    \begin{enumerate}
      \item Subject to dismissal.
      \item Legal action.
    \end{enumerate}
  \end{enumerate}
\end{enumerate}

\subsection{Policy review and modification}

\begin{enumerate}
  \item Scheduled review of policy modifications for modification
  \begin{enumerate}
    \item This organization will actively seek to be in the front of information security, to ensure we are able to offer our services with satisfaction. This policy is therefore an iterative peace of document which is subject to alteration if needed.
  \end{enumerate}
\end{enumerate}

\subsection{Limitations of liability}

\begin{enumerate}
  \item Statements of liability
  \begin{enumerate}
    \item This organization is not liable if an employee does not comply with our policy.
    \begin{enumerate}
      \item The organization will assist in prosecution if necessary.
    \end{enumerate}
  \end{enumerate}
\end{enumerate}


\section{Risk assessment}

\subsection{Unauthorized access to confidential data}

\textbf{Description}: We may store information about our partners and others we offer services to. This data can be discovered security flaws or other compromising information. Unauthorized individuals may try to steal or view this information. Meanwhile, this data is vital for our organization to perform our services. 

\textbf{Impact}: Our organization rely on our reputation. Therefore, the impact of losing confidential information is critical and in a worst case scenario put us out of business.

\textbf{Likelyhood}: As this information is one of our most important assets, it is a target for people with mischievous intent. Therefore, the likelyhood of someone trying to gain access and control over this information is great.

\textbf{Verdict}: Access to confidential data must be authorized, authenticated and logged. It must be stored encrypted on a safe server. In processing it must also be encrypted and never be transferred over unsecured links in plaintext. For telecommuters it is required to have an established VPN tunnel on secure equipment only intended for work.

 %ecryptedn Structure:
%
% REPEAT FOR EVERY, RISK
% \subsection{Title of risk}
% \textbf{Description}: description....
% \textbf{Impact}: impact.....
% \textbf{Likelyhood}: likelyhood.....
% \textbf{Verdict}: verdict.....

\subsection{Branch loosing connection to HQ}

\textbf{Description}: Each branch is reliant on a WAN connection to HQ for services and management resources. This connection is made through a VPN tunnel.

\textbf{Impact}: Staff would loose access to all internal services, and networking equipment would loose access to management resources at the HQ. Staff and guests would still be able to browse the internet like normal.

\textbf{Likelyhood}: We have redundant connections at the HQ, but each branch only has a single gateway router. This router is bound to fail every once in a while, and the connection will break with it. The connection could break the HQ end as well, and this a much more serious problem, but this should be very unlikely compared with the connection breaking at the branch end.

\textbf{Verdict}: This is an accepted risk. To solve this we would have to implement redundancy at each of our 18 branch locations, which would be costly. The problem is addressed at the HQ however, as this is the main reason we have redundant gateways set up there.


\subsection{Leakage of confidential information by employee or volunteer}

\textbf{Description}: Since we rely on hired professionals and volunteers who must have some level of access to perform their work a risk is leakage.

\textbf{Impact}: We handle confidential data about our clients, so a leak would be catastrophic.

\textbf{Likelyhood}: All employees are screened before they get access to the organizations resources. In addition, employes are on a need-to-know basis, so they only have access to resources related to projects they are involved in. This decreases the likelyhood somewhat, but as explained in the verdict we feel this problem should be addressed at a higher level.

\textbf{Verdict}: Network security measures like AAA and logging are in place to support mitigating the issue, but measures at higher levels than that defined by this report are needed, so we judge it to be outside our scope to address this fully.

\subsection{DDoS on our website}

\textbf{Description}: Our role as a information security organization could make us a potential target for DDoS.

\textbf{Impact}: A DDoS would be targeted at our public website. This website going down means potential loss of incoming through donations and purchases in the webshop, but we deem this not to be a huge issue because we are a non-profit organization, and wouldn't miss out on much due to a little downtime. The DDoS would put a strain on our gateway networking equipment, however, meaning internal services would be impacted.

\textbf{Likelyhood}: We deem this to be fearly likely, as we are a public organization focused around infromation security.

\textbf{Verdict}: This is an issue we have to address. Because the only publicly reachable resource is our website, and since that website does not contain any of our client's confidential information, we could set in place a process to bring our website up on a public cloud provider. This is beyond the scope of this project, but worth mentioning. 


\subsection{Failure due to natural disaster at the HQ}

\textbf{Description}: A natural disaster could bring down our infrastructure.

\textbf{Impact}: The HQ going down due to natural disaster would likely mean a halt in all activity at both HQ and all branch sites. In addition we could loose data due to physical destruction of resources.

\textbf{Likelyhood}: Not very likely, as we will place the HQ in a safe location.

\textbf{Verdict}: The impact would be so catastrophic that we have to address this despite the low likelyhood. Essential services and management systems will be placed in racks that can take a beating\cite{todo}, and we will implement backup routines to make sure physically loosing the resources at HQ does not mean loosing essential data.

\subsection{Failure due to power outage at branch site}

\textbf{Description}: A power outage at a branch site is a possbible event.

\textbf{Impact}: A power outage means all the networking equipment at the branch will go down. All activity at the branch will likely halt until the issue is fixed.

\textbf{Likelyhood}: Fearly likely.

\textbf{Verdict}: We could have set up UPS batteries to mitiage the issue, but as everything important is stored at HQ we decide to save costs by not doing this. A UPS system would also only provide us with a little bit of time before we would run out of power anyways, so the branch would go down either way.

\subsection{Failure due to power outage at HQ}

\textbf{Description}: A power outage at a branch site is a possbible event.

\textbf{Impact}: A power outage means that all networking equipment at the HQ would go down. This effectively means that all activity at the HQ would halt, and all activity involving internal resources would halt at branch sites. The impact would be huge.

\textbf{Likelyhood}: Fearly likely

\textbf{Verdict}: We mitigate this by installing UPS batteries for the services and essential networking equipment. This way services could shut down safely in the event of a power outage, or even operate as normal through a short amount of time.

%\notetous{Following was copied from docs:}

%One of the biggest risks is the theft, loss or unauthorized access of confidential data. Especially data that may be stored about partners and others we offer services to.
%\\
%\\
%Since we rely on hired professionals and volunteers which must have some level of access to perform their work. A possible risk may be infiltration and espionage as well as theft of assets under the disguise of altruism.
%\\
%\\
%Others may also try to gain physical unauthorized access to our and partner premises.
%\\
%\\
%Our role in the public can make us a target for Distributed Denial of Service attacks (DDoS).
%\\
%\\
%As everyone else we are at risk of human error and/or failure without mischievous or criminal intent. 
%\\
%\\
%Power outage, natural disasters.

\section{ICT and Network Infrastructure}

% TOPICS TO DISCUSS, IN AN INITIAL ORDER:
%
% * include logical and physical view, and disucss briefly how to view the diagram. used in comming sections (?)
% * talk about WAN connection
% * Talk about subnets and VLANs and so on
% * talk about wireless and so on
% * talk about lab area
% * Branch specific stuff
%   * dhcp on the router
%   * firewall on the router
% * HQ specific stuff
%   * redundancy, loadbalancing (glbp)
%   * services area, redundancy there etc
%   * what goves over WAN
%   * "remote commuting" (if that's what it's called)

\notetous{Short flavortext about what will be discussed}

% make sure to emphasize that the goal was cost-savings


\subsection{Layout and Diagrams}

\notetous{Show logical and phyical views, and provide some discussion around them. The diagrams included are only temporary, proper ones will be added later.}

The following digrams shows a few different views of our network infrastructure. Short explanations of the diagrams are included.

\subsubsection{Logical view}

%Figure \ref{logicalview} shows a logical view of our network infrastructure. We will explain the choices we made in detail later in this report, but we feel we should make some points about the diagram itself. 

The diagram in Figure \ref{logicalview} shows a view of our infrastructure. We've put the routers inside the distribution system to show that we have chosen a collapsed core solution. Note that the "distribution system" on the branch site really just consists of a single router and a few switches, but we felt the terminology still fit. The "services and management resources" area is placed above the distribution system to indicate that this is given it's own dedicated switches. The lab has it's own access layer area in the diagram to indicate that it will be given special treatment.

\begin{figure}[H]
\caption{Logical view of the network}
\centering
\includegraphics[width=\textwidth]{logicalview}
\label{logicalview}
\end{figure}

\subsubsection{Physical view}

We've used screenshots of our Packet Tracer implementation to produce the diagrams seen in Figure \ref{physicalviewhq} and \ref{physicalviewbranch}.

\notetous{More info around diagrams? idk if there is more to say}

% TODO: update the physical diagrams

\begin{figure}[H]
\caption{Physical view of HQ (TEMPORARY)}
\centering
\includegraphics[width=\textwidth]{physicalviewhq}
\label{physicalviewhq}
\end{figure}

\begin{figure}[H]
\caption{Physical view of Branch (TEMPORARY)}
\centering
\includegraphics[width=\textwidth]{physicalviewbranch}
\label{physicalviewbranch}
\end{figure}


\subsection{WAN}

For the site to site WAN between HQ and branches, we have implemented an IPsec tunnel that traverses the internet, in order to connect branches and HQ without having to pay for a leased line, which is too expensive for us and therfore the reason to NOT use PPP or PPPoE, that cannot be transmitted over the internet as they are L2 protocols. IPsec on the other hand is a L3 protocol and therefore can be transmitted over the internet, which is a very affordable medium, also safe with IPsec encryption and we also don't really need the bandwidth of a leased line.

We also want to be able to VPN to HQ from an off-site location, so we could have a VPN server with a pool of temporary IP addresses for remote locations. This VPN would pass through an IPsec tunnel forwarded by the ISP, to an off-site client.

\subsection{VLAN's and Subnets}

The goal when designing our VLAN's and subnets was simplicity. We chose to subnet our private addresses from the 10.0.0.0/8 private address range, as it gave us plenty of space to the subnets up how we wanted.
We dedicate the second octet to indicate what location the subnet belongs to, where the value 1 indicates the HQ, and a value above 1 indicates a branch site. For example: 10.1.x.x is a subnet the HQ, and 10.2.x.x is a subnet at the first learning centre and so on.

This lets us create subnets that look very similar across locations, while letting a network engineer quickly see what location a particular IP address belongs to. We dedicate the third octet to indicate what VLAN the subnet belongs to, with the value of the octet being the same as the VLAN id. 10.1.10.x is a subnet for the management vlan of the HQ, and 10.10.60.x is a subnet for the printer VLAN of location nr 9. Combined with the fact that we reserve the first 10 hosts for static IP's in each subnet range, this leaves us with 244 available dynamic host addresses.

For the learning centres this is not a problem, as 244 guests are unlikely to be connected to the network at the same time. The HQ is more brittle, but we've condluded that 244 hosts should be enough there as well. Table \ref{vlansubnettable} shows our VLAN's with their corresponding subnets.

% TODO: 10.0.x.x = ipsec (subnet bruk mellom WAN linker)

% FIXME: H necessary?
% Forces table to show up under text
\begin{table}[H]
\caption{Table of VLANs with corresponding subnets}
\label{vlansubnettable}
\begin{tabular}{|l|l|l|l|l|}
\hline
\textbf{VLAN Name} & \textbf{VLAN ID} & \textbf{Subnet} & \textbf{Excluded addresses} & \textbf{Available addresses} \\ \hline

Management     & 10      & 10.x.10.0/24 & 10.x.10.1 - 10.x.10.10 & 244 \\ \hline
Staff          & 20      & 10.x.20.0/24 & 10.x.20.1 - 10.x.20.10 & 244 \\ \hline
Services       & 30      & 10.x.30.0/24 & 10.x.30.1 - 10.x.30.10 & 244 \\ \hline
Lab            & 40      & 10.x.40.0/24 & 10.x.40.1 - 10.x.40.10 & 244 \\ \hline
Guest          & 50      & 10.x.50.0/24 & 10.x.50.1 - 10.x.50.10 & 244 \\ \hline
Printer        & 60      & 10.x.60.0/24 & 10.x.60.1 - 10.x.60.10 & 244 \\ \hline
DMZ            & 70      & 10.x.70.0/24 & 10.x.70.1 - 10.x.70.10 & 244 \\ \hline
Blackhole VLAN & 99      & N/A          & N/A                    & N/A \\ \hline
\end{tabular}
\end{table}

\subsection{Wireless}

% TODO: should use something other than PSK in the real setup?? google

We've chosen to build our wireless solution with Wireless Access Controllers (WLC's) and Lightweight Access Points (LWAP'S). As we want as few resources as possible to be placed at the branch locations, the WLC's will be placed at the HQ. We will have redundancy in WLC by the use of the "High Availability" features that Cisco provides.
\\
\\
This means that the LWAP's at the branches might encounter situations with no access to WLC's, but as this connection will benefit from the same redundancy as the one provided for services this will be relatively rare. We'd rather put resources into our one HQ than to spread resources around to each of the 18 branches.
\\
\\
The WLC's and LWAP's use CAPWAP to communicate, meaning all LWAP's are connected to access ports on the management VLAN. This unifies how we set up our AP's.
\\
\\
Our Wireless setup consists of two WLAN's, one for staff and one for guests. Both are configured with WPA2 level security. The staff WLAN is tagged to the staff VLAN, and the guest WLAN is mapped to the guest VLAN, this way staff and guest users connected over Wi-Fi will have the same rights as those connected physically.

\subsection{Lab}

%\notetous{Discuss how we implemented the lab environment}

The purpose of the labs was to provide an environmet where visitors could experiment freely with information security concepts. This means that it is a security risk by design, so we have to be careful how we implement it. The labs will have their own physical switch, and the kinds of traffic that may pass through will be heavily restricted. Only browsing is allowed, so we allow outgoing connections on port 80 and 443 (WHAT ELSE THOUGH? KEMMERICH SAID THIS WASN'T ENOUGH). 
\\
\\
Some networking equipment will be placed in the lab to be used for experimentation, but we don't consider this a part of our network infrastructure.

% TODO: what traffic to allow? not just 80 443
% TODO: clean up plural vs singular
% TODO: elaborate

\subsection{Branch network discussion}

% TODO more sections?
%\notetous{Discuss topics specific to the branch network setup}

As seen in Figure \ref{physicalviewbranch}, the branch network is very simple. As all resources are located at the HQ, each branch only needs a router, a few switches and a few LWAP's to function. Each branch site will not have much traffic, so we can justify saving costs by placing the firewall, local DHCP server and the VPN termation at the router itself.

The LWAP's are configured through CAPWAP tunnels going to the WLC's at HQ. We could have set up redundant WLC functionality on each branch site, but decided against it because the cost of configuring this for each branch would be too high.
\\
\\
We don't consider the branch site as critical infrastructure. Rather than providing each branch site with redundancy, we place all resources at the HQ, and focus on providing redundancy there. This means that a branch might go down for a few days every once in a while. We consider this an accepted risk.

\subsection{Head Quarters network discussion}

\subsubsection{Layout}

Initially we aimed for a three-tier architecture~\cite{todo}, but found out that we a collapsed-core architecture~\cite{todo} heavy enough. A collapsed-core architecture provides similar efficiency benefits as the three-tier architecture, but with a focus on saving costs.

We have two routers and two switches set up to combine the core and distribution layer into a single layer. The services and management resources are given special treatment and placed in it's own segment of the network, while the rest of the network is set up to use access switches with connections to both distribution/core layer switches to provide redundancy.
\\
\\
Phyically each access layer switch is configured with access ports based on what users are found around the switch. Switches placed in the offices will have staff and management ports, while switches in lecture halls and such will have guest ports. The switches LWAP's are connected need to be management ports. 

% TODO: elaborate?

\subsubsection{Redundancy}

As all of the branch sites rely on the HQ to operate, redundancy was an important topic to address. What really needs redundancy are the services and management resources. The fact that the rest of the HQ gets some redundancy is really just a bonus.

As shown in Figure~\ref{physicalviewhq}, the HQ has two links to the ISP through a duplicated setup of routers and switches. Important services and management resources are placed in the middle of the network, in between the two routers, as shown. They have dual ethernet connections, one going to each ISP link. This way, if any of the equipment forming one of the connections goes down, the services and management resources will still have internet connectivity.
\\
\\
As we have have two gateways we need to use a "First Hop Redundancy" protocol to manage the deafult gateway. The options we looked at were HSRP, VRRP and GLBP. We went with GLBP because in addition to providing the redundancy we need, it also does load balancing, letting us utilize both of the gateway routers to their fullest. HSRP and VRRP would've provided redundancy, but not load balancing.


% TODO add note about use of HSRP in demo implementation?
% TODO refer to diagrams!!!!!!!!!! "as seen in \ref{...}" and so on

% not needed? talked about in above subsection
%\subsubsection{Services}
%\notetous{Discuss how services are handled at HQ}

\subsection{Methods for hardening}

\subsubsection{Layer 2}
As layer 2 can be the weakest link in our network infrastructure we will implement different methods to prevent or mitigate potential attacks:

\textbf{CAM table attack}: To prevent flooding of a switch's CAM table all access ports have port-security implemented. This also prevents DHCP starvation.

\textbf{VLAN hopping and double-tagging attack}: To prevent an attacker from spoofing a switch trunk all ports have disabled DTP(auto trunking). Ports available to users are explicitly assigned as access ports. To prevent a double-tagging attack trunks have a native BLACKHOLE VLAN. Ports which are not in use are disabled. They are also configured as access ports to the BLACKHOLE VLAN to force awareness of the configuration when turned back on.

\textbf{Management traffic}: Whereas Lab have a separate switch, management and staff share the same switch. For additional security all access ports on this switch use PVLAN edge to make it difficult for a man-in-the middle attack against management traffic.

\textbf{DHCP, ARP and IP spoofing attacks}: To prevent rogue DHCP servers and DHCP starvation we will use DHCP Snooping. While our upstream ports will be set to trusted, access port will have a limit to how many DHCP request it can send per second. However, it may take some time for the DHCP Snooping to finish the binding table. In interaction with DHCP Snooping we will use DAI (Dynamic ARP Inspection) to prevent ARP spoofing and poisoning. Then the switch relay only valid ARP replies. In addition we will implement IPSG (IP Source Guard) to prevent IP address spoofing.


\subsection{Our demo Packet Tracer implementation}

% make sure to include that:
% * HSRP instead of GLBP
% * wireless not configured in packet tracer
% * list of what is implemented and what isn't?
% * explain how we simplified WAN for the demo

As part of the project we implemented part of our infrastructure in Packet Tracer. The configuration can be found in appendix \ref{config}. This section will provide some discussion around the demo implementation.

\subsubsection{HQ demo configuration}

Figure \ref{physicalviewhq} is taken from our PT implementation of the HQ. As you can see we have configured three access layer switches with a few machines attached to different access ports. The services and management resources area is set up with it's own dedicated switches connected to the "distribution system", which is really just two switches connected to two gateway routers. The left gateway router shows how defence-in-depth would be implemented, and also does our VPN connection to the branch site. The firewall is not plugged in, and the DMZ is not set up. In reality the firewall would be placed between the two routers, and traffic from the public web would be routed to a DMZ connected to the firewall, as illustrated by Figure \ref{logicalview}. The setup would be duplicated on the right gateway router.
\\
\\
In reality we would like to use GLBP for "first hop redundancy" and load balancing, but we did not figure out a way to do this in Packet Tracer. For demonstration purposes we configured HSRP on the routers instead.
\\
\\
The WLC would in reality be coupled with at least one standby controller, but we've only included one for the demo. It is plugged in, but does nothing. See \ref{demowireless} for further discussion on this topic.

% TODO: add lwaps to PT
% TODO: add refs to appendix configs
\notetous{Johannes could add notes about security measures that were implemented?}

\subsubsection{Branch demo configuration}

As all the branches have similar configuration, we only implemented one in our demo. Figure \ref{physicalviewbranch} shows our implementation. % <- TODO: REWRITE
The router is very similar to how it would be configured in reality, the only different being that a firewall has not been set up. The LWAP's connected to the main switch are intended to be placed around the building, and would get their configuration from the WLC at HQ.

\subsubsection{Wireless demo configuration} \label{demowireless}

We had trouble with implementing WLC's and LWAP's properly in Packet Tracer, so we decided configure a physical WLC as a proof-of-concept of how the WLC at HQ in our demo would look like. The exported configuration is provided in appendix \ref{configwlc}. Two WLAN's Staff and Guest are set up with WPA2+PSK security. Note that the addresses used in the configuration file are wrong. All addresses starting with 192.168 should really start with 10.1. We configured it like this because we had trouble with the 10.0.0.0/8 range in the cisco lab.

\section{Conclusion}

\notetous{An amazing conclusion that will knock the socks of Ernst boiiiiiiiiiiiiiiiiiii}

% BEGINNING OF REFERENCES

\clearpage % make references start on own page

% This makes LaTeX list all references in the bib file, instead of just 
% the ones that are cited with a \cite command
\nocite{*}

% BEGINNING OF APPENDIX

\bibliographystyle{acmdoi}
\bibliography{report}

\clearpage % make appendix start on own page
\appendix

\section{Our demo Packet Tracer implementation} \label{config}

\notetous{TODO: Add configuration files here.}

\subsection{(Physical) WLC configuration} \label{configwlc}

\inputminted[fontsize=\tiny,linenos,breaklines]{text}{./wirelessimplementation/wlc.cisco}


\end{document}
