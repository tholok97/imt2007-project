\input{preamble.tex}

\title{A template \LaTeX{} report}
\author{Author Authorsen}

\begin{document}

\maketitle

\abstract{This document demonstrates usage of \LaTeX{}.THOLOOOK}

\thispagestyle{empty}

\clearpage
\pagenumbering{roman}
\setcounter{page}{1}
\tableofcontents

\clearpage
\pagenumbering{arabic}

% BEGINNING OF CONTENT

\section{Business case}

SikreNorge is a non-profit charity organization that provides information security services to businesses and institutions in Norway. The organization seeks to secure Norway by helping those who would not otherwise have the means to prioritize security.

Financially SikreNorge relies on donations, government subsidies, and income from their web-shop. Businesses who donate are encouraged to publicise their contribution to show that they support the organization's mission. This way both the contributing businesses and SikreNorge gets good PR. Government subsidies are given with the intention and expectation that the organization will use these means to support local public and private institutions with information security tasks.

The organization is reliant on local expertise in each county in the form of volunteers and hired professionals. Volunteers are expected and encouraged to gain security competence through the experts. Good candidates for volunteers are students doing technical studies, it could however be anyone. Where possible the organization will make agreements with local universities teaching information technology courses to encourage students to volunteer. The organization is also open to volunteer work as part of theses and course projects. Volunteers of the organization do work in exchange for practical information security experience. Both volunteers and hired professionals are expected to sign strict non-disclosure agreements regarding their work with businesses and institutions.

\subsection{Locations}

SikreNorge will establish learning centers in each of Norway's 19 counties. The headquarter lies in Oslo.

Learning centers are meant as a hub for all the organizations activities in the applicable county. The centers has offices for the local management and the hired professionals. They also contain network sandbox areas for practicing network security(Virtual and otherwise), classrooms for security courses(Possibly lecture halls) , lecture halls(could also use classrooms or other offices) for seminars and general storage space. In addition to activities held at the centers, the organization can send consultants into the field to do evaluation or configuration on-site. Consultants are a mix of local hired professionals, volunteers and staff.

The organization's headquarter provides the same services as learning centers, but also does the heavy lifting in terms of information technology services and infrastructure. The website, E-mail and management applications and all the large equipment too which one can connect from another center for data resources; are examples of services that are hosted here.

\subsection{Services}

The organization includes a wide range of services, and the network must be set up to support these. Following is a brief discussion of some key services, along with some infrastructure concerns.

\subsubsection{On-site company consultation}

Volunteers, hired professionals and staff  can be sent out to do consulting on-site at businesses or institutions. Access to the organizations internal services might be required during consultation, so the consultants should be able to reach them from on-site. Confidential information might be sent and received during this communication, so the technology chosen for this should focus on security.
Teaching at learning centre
Businesses and institutions may visit a learning centre to take courses or to receive guidance. During this stay they will naturally want internet connectivity. In addition, access to some internal services might be needed for training purposes. A visitors access rights should be restricted as much as possible.

\subsubsection{Network security laboratory}

Each learning centre has a laboratory where visitors can explore hardware and software to learn about vulnerabilities and best practises. Labs are meant to be a sandbox area for learning about information security, so they should be separated from the rest of the distribution system in such a way that experiments from the lab cannot interfere with the rest of the learning centers operations.

\subsubsection{Public website}

SikreNorge's website includes information about the organizations offerings, online courses and a web-shop. Some of the website's information and courses might build upon internal services, so the infrastructure should support communication to allow this. The web-shop will sell T-shirts, coffee mugs, posters and more to further information security awareness and good practise. The shop allows buyers to register accounts for easy shopping.

\subsubsection{Management website}

The organization uses a website separate to the public website to manage internal logistics. Registered staff, businesses, institutions and overview of future work are examples of information that can be found here.

\section{Security Policy}
\section{Risk}
\section{ICT}
\section{Conclusion}

\clearpage

\textit{Everything beyond this point is from the template}

\section{Introduction}

This is a \LaTeX{} report template. It demonstrates cross-references, citations, images, and code examples with minted.

\section{Examples}

\subsection{Code}

The following is a demonstration of how cross-references can be used to refer to appendix code. Citations are also used.
\\
\\
Beaker can work with different hypervisors by using plugins. If a plugin for a particular hypervisor does not exist, an alternative is to use Vagrant to manage the SUT's, and instead install and run Beaker as part of Vagrant's provisioning process.~\cite{puppetmodulefunctionaltestingbeakerinsidevagrant}\cite{openstackpuppetmodulefunctionaltestingproposal} An example of this is included in appendix~\ref{beakerinsidevagrantexample}. This example would be used by running \mintinline[breaklines]{bash}{vagrant up --verbose && vagrant destroy --force --verbose}.

\subsection{Images}

Figure \ref{sunnmore} shows a picture of Sunnmørsalpane. You're welcome.

\begin{figure}[H]   % H forces the picture to show up under the text
\caption{Sunnmørsalpane}
\centering
\includegraphics[width=\textwidth]{sunnmore}
\label{sunnmore}
\end{figure}

\section{Conclusion}

This has been a demonstration of \LaTeX{} in use.

% BEGINNING OF REFERENCES

\clearpage % make references start on own page

% This makes LaTeX list all references in the bib file, instead of just 
% the ones that are cited with a \cite command
\nocite{*}

% BEGINNING OF APPENDIX

\bibliographystyle{acmdoi}
\bibliography{report}

\clearpage % make appendix start on own page
\appendix

\section{Beaker inside Vagrant example} \label{beakerinsidevagrantexample}

\begin{minted}[fontsize=\tiny,linenos,breaklines]{ruby}
# -*- mode: ruby -*-
# vi: set ft=ruby :

require 'vagrant-openstack-provider'

#
# This is quite the minimal configuration necessary
# to start an OpenStack instance using Vagrant on
# an OpenStack with Keystone v3 API
#
# NOTE: this example is heavily
# inspired by http://my1.fr/blog/puppet-module-functional-testing-with-vagrant-openstack-and-beaker/
#
Vagrant.configure('2') do |config|

  config.ssh.username = 'ubuntu'

  config.vm.provider :openstack do |os, ov|
    os.server_name                      = 'vagrant_machine_in_openstack'
    os.security_groups                  = [ 'default', 'linux' ]
    os.identity_api_version             = '3'
    os.openstack_auth_url               = 'https://api.skyhigh.iik.ntnu.no:5000/v3'
    os.project_name                     = '<PROJECTNAME>'
    os.user_domain_name                 = 'NTNU'
    os.project_domain_name              = 'NTNU'
    os.username                         = '<USERNAME>'
    os.password                         = '<PASSWORD>'
    os.region                           = 'SkyHiGh'
    os.floating_ip_pool                 = 'ntnu-internal'
    os.floating_ip_pool_always_allocate = true
    os.flavor                           = 'm1.small'
    os.image                            = 'Ubuntu Server 16.04 LTS (Xenial Xerus) amd64'
    os.networks                         = [ '<INTERNALNETID>' ]

    ov.nfs.functional = false
  end

  # you could provision this machine using the same provisioning scripts used by 
  # Heat, to create an exact duplicate
  config.vm.provision "shell", path: "bootscriptFromHeat.sh"
  

  # shell to install beaker, setup ssh, and run beaker tests.
  # written inline for sake of example
  config.vm.provision "shell", inline: <<-SHELL
    #!/bin/bash

    # install deps
    sudo apt-get update
    sudo apt-get install -y libxml2-dev libxslt-dev zlib1g-dev git ruby ruby-dev build-essential

    # prepare ssh
    echo "" | sudo tee -a /etc/ssh/sshd_config
    echo "Match address 127.0.0.1" | sudo tee -a /etc/ssh/sshd_config
    echo "    PermitRootLogin without-password" | sudo tee -a /etc/ssh/sshd_config
    echo "" | sudo tee -a /etc/ssh/sshd_config
    echo "Match address ::1" | sudo tee -a /etc/ssh/sshd_config
    echo "    PermitRootLogin without-password" | sudo tee -a /etc/ssh/sshd_config
    mkdir -p .ssh
    ssh-keygen -f ~/.ssh/id_rsa -b 2048 -C "beaker key" -P ""
    sudo mkdir -p /root/.ssh
    sudo rm /root/.ssh/authorized_keys
    cat ~/.ssh/id_rsa.pub | sudo tee -a /root/.ssh/authorized_keys
    sudo service ssh restart
   
    # prepare gems
    # this uses my gossinbacup module as an example, but it would be
    # possible to have the module as a parameter to this process
    git clone https://github.com/tholok97/gossinbackup
    cd gossinbackup
    sudo gem install bundler --no-rdoc --no-ri --verbose
    bundle install

    # run tests
    # this relies on SUT yaml definitions with hyporvisor set to "none",
    # like here: https://github.com/openstack/puppet-keystone/blob/master/spec/acceptance/nodesets/nodepool-xenial.yml
    export BEAKER_debug=yes
    bundle exec rspec spec/acceptance
SHELL

end
\end{minted}

\end{document}
